% ----- CHAPTER 2: PROBLEM OUTLINE AND RESULTS ----- %

A natural question to ask in the field of computational number theory is: does an algorithm exist to compute the rank of an elliptic curve? Manin showed in \cite{Man-1971} that, contingent on the Birch and Swinnerton-Dyer Conjecture being true, the answer to this question is yes. A rough outline of the method delineated by Manin is as follows: by day you search for points on a curve and thus obtain a lower bound on the algebraic rank of the curve; by night you evaluate central derivatives of the curve's $L$-function and thus obtain an upper bound on the analytic rank. If the Birch and Swinnerton-Dyer Conjecture is true, then eventually the two bounds will match up, and you will have computed the curve's rank. \\

However, although conjecturally guaranteed to terminate, the above method is ineffective from a time complexity perspective -- there are no results establishing just how long it will take for the two bounds to match up. It is thus a somewhat less than satisfying answer to the question posed. \\

We therefore modify the question to the following: given a rational elliptic curve $E$, does an algorithm exist to compute the rank of $E$ {\it that has provable big-Oh runtime in some measure of the arithmetic complexity of the curve}? In this work we answer this question in the affirmative. \\

The relevant measure of arithmetic complexity is the conductor $N_E$; below we provide an algorithm to compute rank and prove that it has polynomial runtime in $N_E$. However, to do so we must pay the price of having to assume not only the Birch and Swinnerton-Dyer Conjecture, but also the ABC Conjecture. The algorithm can be further sped up by assuming the Generalized Riemann Hypothesis. From hereon out we refer to these three big conjectures as BSD, ABC and GRH respectively. \\

Specifically, we establish the following result:
\begin{theorem}\label{thm:main_theorem}
Let $E/\QQ$ have conductor $N_E$. Contingent on BSD and ABC, there exists an algorithm to compute the algebraic rank $r_E$ of $E$ in $\softO\left(\sqrt{N_E}\right)$ time and $\softO(1)$ space. That is, for any $\epsilon>0$ the algorithm is guaranteed to terminate with a correct output in $O\left((N_E)^{\frac{1}{2}+\epsilon}\right)$ time, using only $O\left((N_E)^{\epsilon}\right)$ space.
\end{theorem}
The algorithm in question is as follows:
\begin{algorithm}{Compute the rank of an elliptic curve}\label{algo:compute_rank}
Given a rational elliptic curve $E$ represented by a global minimal Weierstrass equation $y^2 + a_1 xy + a_3 y^2 = x^3 + a_2 x^2 + a_4 x + a_6$ with known conductor $N_E$:
\begin{steps}
\item Compute the real period $\Omega_E$ of $E$.
\item Set $k = 65 + 10.5 \log_2 N_E - \log_2 \Omega_E$, and set $m=0$.
\item Evaluate $\frac{L_E^{(m)}(1)}{m!}$, the $m$th Taylor coefficient of the $L$-function of $E$ at the central point, to $k$ bits precision. If all $k$ bits are zero, increment $m$ by 1 and repeat this step.
\item Output $r_E=m$ and halt.
\end{steps}
\end{algorithm}

Furthermore, if one also assumes GRH, step 2. can be replaced with:
\begin{enumerate}
\item[2.] Set $k = 31 + 6.7 \log_2 N_E - \log_2 \Omega_E$, and set $r=0$.
\end{enumerate}
This will reduce the runtime of Algorithm \ref{algo:compute_rank} by a constant factor. \\

This algorithm is not new -- it is just a refinement on bounding the analytic rank of a curve with an explicitly chosen precision. What {\it is} new is the body of results in this dissertation proving that, assuming BSD and ABC (and optionally GRH), if the $m$th derivative of the $L$-series attached to $E$ is zero to $k$ bits precision, then it {\it is} identically zero. This allows us to convert an algorithm that a priori only provides upper bounds on analytic rank, to one that computes rank exactly. Furthermore, we show that the algorithm is guaranteed to terminate in time polynomial in the curve's conductor. \\

Moreover, Algorithm \ref{algo:compute_rank} is in a sense optimal among analytic rank computation methods: since evaluating $\Les$ takes $\tilde{O}(\sqrt{N_E})$ time, we cannot hope to get the rank out in time faster than this. [Of course other rank computation methods do exist, but they don't involve working with $\Les$ directly.] The proof of Theorem \ref{thm:main_theorem} can be found in section \ref{sec:main_thrm_proof}, but will require results established in preceding and following sections. \\

Along the way, we also prove the following interesting results regarding bounds on the locations of the nontrivial zeros of the $L$-function attached to $E$:

% QUOTE MAJOR RESULTS