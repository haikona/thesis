\documentclass[10pt]{article}
\include{macros}

\usepackage{fullpage}
\usepackage{hyperref}
\usepackage{microtype}
\usepackage{mathtools}
\usepackage[usenames,dvipsnames]{color}
\usepackage[pdftex]{graphicx}

\usepackage{amsmath}
\usepackage{amssymb}
\usepackage{url}


\newcommand{\NN}{\mathbb{N}}
\newcommand{\B}{\mathcal{B}}
\newcommand{\qedb}{\hfill $\blacksquare$}
\newcommand{\nrm}[1]{\left\|#1\right\|}
\newcommand{\pr}{^{\prime}}
\newcommand{\Les}{L_E(s)}
\newcommand{\Lfs}{L(f,s)}
\newcommand{\ldLes}{\frac{L_E\pr}{L_E}(s)}
\newcommand{\ldLfs}{\frac{L_f\pr}{L_f}(s)}
\newcommand{\ldLe}[1]{\frac{L_E\pr}{L_E}\left(#1\right)}
\newcommand{\ldLf}[1]{\frac{L_f\pr}{L_f}\left(#1\right)}
\newcommand{\ldLam}[1]{\frac{\Lambda_E\pr}{\Lambda_E}\left(#1\right)}
\newcommand{\ldatzero}{\frac{L\pr(E,0)}{L(E,0)}}
\newcommand{\Ensfpe}{E_{\text{ns}}(\FF_{p^e})}


\title{Ph.D Dissertation}
\author{{\bf Simon Spicer} \\ University of Washington}
\date{\today}
\begin{document}
\maketitle

%%%%%%%%%%%%%%%%%%%%%%%%%%%%%%%%%%%%%%%%%%
\section{Introduction}

This is a check to see if we can successfully cite! \cite{Bob-2011}

Here is another line of text. We add some more text to check if the .tex file can compile locally. Cheese.

\section{Bounding the Regulator}

To define the regulator of a rational elliptic curve, we must first define the na\"ive logarithmic height, N\'eron-Tate canonical height and the N\'eron-Tate pairing on points on $E$.

Let $E$ be an elliptic curve over $\QQ$ and $P \in E(\QQ)$ a rational point on $E$. 

\begin{definition}
The {\it na\"ive logarithmic height} of $P$ is a measure of the ``complexity'' of the coefficients of $P$. Specifically, any non-identity rational point $P$ may be written as $P = (\frac{a}{d^2},\frac{b}{d^3})$, with $a,b,d \in \ZZ$, $d>0$ and $\gcd(a,b,d) = 1$; we then define the na\"ive height of $P$ to be
\begin{equation}
	h(P) := \ln(d).
\end{equation}
Moreover, define $h(\cO) = 0$.
\end{definition}
If you compute the na\"ive heights of a number of points on an elliptic curve, you'll notice that the na\"ive height function is ``almost a quadratic form'' on $E$. That is $h(nP) \sim n^2 h(P)$ for integers $n$, up to some constant that doesn't depend on $P$. We can turn $h$ into a true quadratic form as follows:

\begin{definition}
The {\it N\'eron-Tate height} height function $\hat{h}: E(\QQ) \to \RR$ is defined as
\begin{equation}
	\hat{h}(P) := \lim_{n \to \infty} \frac{h(2^n P)}{(2^n)^2},
\end{equation}
where $h$ is the na\"ive logarithmic height defined above.
\end{definition}

\begin{theorem}[N\'eron-Tate]
N\'eron-Tate has defines a canonical quadratic form on $E(\QQ)$ modulo torsion. That is,
\begin{enumerate}
	\item For all $P,Q \in E(\QQ)$,
	\begin{equation}
		\hat{h}(P+Q) + \hat{h}(P-Q) = 2\left[ \hat{h}(P) + \hat{h}(Q)\right]
	\end{equation}
	i.e. $\hat{h}$ obeys the parallelogram law;
	\item For all $P \in E(\QQ)$ and $n \in \ZZ$,
	\begin{equation}
		\hat{h}(nP) = n^2 \hat{h}(P)
	\end{equation}
	\item $\hat{h}$ is even, and the pairing $\langle\;,\;\rangle: E(\QQ)\times E(\QQ) \to \RR$ by
	\begin{equation}
		\langle P,Q \rangle = \hat{h}(P+Q) - \hat{h}(P) - \hat{h}(Q)
	\end{equation}
	is bilinear;
	\item $\hat{h}(P) = 0$ iff $P$ is torsion;
	\item We may replace $h$ with another height function on $E(\QQ)$ that is ``almost quadratic'' without changing $\hat{h}$.
\end{enumerate}
\end{theorem}

For a proof of this theorem and elaboration on the last point, see \cite[pp. 227-232]{Sil-1985}.

\begin{definition}
The {\it N\'eron-Tate pairing} on $E/\QQ$ is the bilinear form $\langle\;,\;\rangle: E(\QQ)\times E(\QQ) \to \RR$ by
	\begin{equation}
		\langle P,Q \rangle = \hat{h}(P+Q) - \hat{h}(P) - \hat{h}(Q)
	\end{equation}
\end{definition}
Note that this definition may be extended to all pairs of points over $\QQbar$, but the definition above suffices for our purposes. \\

If $E(\QQ)$ has rank $r$, then $E(\QQ)/E_{\text{tor}}(\QQ) \hookrightarrow \RR^r$ under the quadratic form $\hat{h}$ as a (rank $r$) lattice.

\begin{definition}
The {\it regulator} $\Reg_E$ of $E/\QQ$ is the covolume of the lattice that is the image of $E(\QQ)$ under $\hat{h}$. That is, if $\set{P_1,\ldots,P_r}$ generates $E(\QQ)$, then
\begin{equation}
	\Reg_E = \det\left(\langle P_i,P_j\rangle \right)_{1 \le i,j \le r}
\end{equation}
where $\left(\langle P_i,P_j\rangle \right)_{1 \le i,j \le r}$ is the matrix whose $(i,j)$th entry is the value of the pairing $\langle P_i,P_j\rangle$. If $E/\QQ$ has rank zero, then $\Reg_E$ is defined to be 1.
\end{definition}

Loosely, the regulator measures the ``density'' of rational points on $E$: positive rank elliptic curves with small regulators have many points with small coordinates, while those with large regulators have few such points. \\

It turns out that one can construct elliptic curves with arbitrarily large regulators (for example, fix some $x_0$ and $y_0$ with large denominators, and then find $A$ and $B$ such that $P=(x,y)$ lies on $E: y^2 = x^2 + Ax + B$). However, the more interesting question to ask -- and the one that is relevant to this thesis -- is ``how small can the regulator get?''. Specifically, given $E/\QQ$ with discriminant $D_E$, what is the smallest $\Reg_E$ can be as a function of $D_E$?

This is an open question. However, the following conjecture has been made by Lang \cite{Lang-1997}:

\begin{conjecture}
Let $E/\QQ$ have minimal discriminant $\Delta_E$. There exists an absolute constant $M_{\QQ}>0$ independent of $E$ such that any non-torsion point $P \in E(\QQ)$ satisfies
\begin{equation}
\hat{h}(P) >= M_{\QQ} \log |\Delta_E| .
\end{equation}
\end{conjecture}
That is, the minimum height of a non-torsion point on $E$ scales with the log of the absolute value of the curve's minimal discriminant. Hindry and Silverman in \cite{HiS-1988} show that the abc conjecture implies Lang's height conjecture; since we are already assuming strong abc, we have this result for free.

A corollary of the height conjecture is that the regulator is bounded from below in terms of the curve's conductor:
\begin{corollary}
Let $E/\QQ$ have conductor $N_E$, and let $M_\QQ$ be the absolute constant in Lang's height conjecture. Then
\begin{equation}
\Reg_E \ge 3
\end{equation}
\end{corollary}
\begin{proof}
Let $E$ have minimal discriminant $\Delta_E$ and algebraic rank $r$. Since $|\Delta_E|\ge N_E$, it follows that 
\begin{equation*}
\Reg_E \ge \left(M_{\QQ} \log N_E \right)^r \ge M_{\QQ}^r
\end{equation*}
\end{proof}









\bibliography{bibliography}{}
\bibliographystyle{amsalpha}

\end{document}
